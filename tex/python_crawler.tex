

\chapter{Python Crawler \small(Stach \& Ilyina)}
\label{ch:Python Crawler}

Scrapy bietet ein CLI Programm zum Erstellen eines initialen Scrapy Projektes.
Das Projekt ist danach logisch in folgende Komponenten aufgeteilt: Spiders, Items, Middlewares, Pipelines.
Im Folgenden gehen wir darauf ein, wie die unterschiedlichen Komponenten arbeiten und wie wir sie für die
Erreichung unseres Lernzieles eingesetzt haben.

\section{Planung}

Um die Implementierung des Python-Crawlers zu planen haben wir uns zuerst
über Libraries in Python zum Crawlen von Webseiten informiert. Die im
Internet meist verwendeten Libraries sind BeautifulSoup4\cite{beausoup}
und Scrapy\cite{scrapy}. Bei BeautifulSoup fiel direkt auf, dass das Projekt
nicht mehr aktiv weiterentwickelt wird, da der letzte Commit auf Github am
05.06.2015 war. Scrapy ist außerdem ein komplettes Framework mit etlichen
bereits vorhadenen Werkzeugen und einer sehr guten Dokumentation. Deswegen
fiel unsere Wahl auf Scrapy.

\section{Implementierung}

\subsection{Spider}

Ein Spider ist der Kern von Scrapy. Dieser crawlt die Seiten und extrahiert gewünschte Informationen.
Es wird dabei eine Start-URL definiert. Ausgehend von dieser URL crawlt Scrapy alles was als Request-Objekt in
eine Queue gelegt wird. Ein Request-Objekt besteht aus der zu crawlenden URL, sowie einer Referenz zu einer Callback-Funktion,
mit der die Seite gecrawlt werden soll. Um Informationen, wie z.B. URLs, aus dem HTML Quelltext zu extrahieren, kann man bei Scrapy
mit CSS-Selektoren arbeiten.

Auf der Seite der Berliner Polizei gibt es drei verschiedene Seitenarten die vom Crawl-Vorgang
unterschiedlich behandelt werden müssen. Die Archivübersichtsseite beinhaltet Links zu den unterschiedlichen Archiven.
Archive sind nach Jahren kategorisiert. Wir haben alle Links zu den Jahresarchiven extrahiert und sie als Request-Objekt in die Queue gelegt.

\begin{minted}[xleftmargin=20pt,linenos,breaklines,fontsize=\small]{python}
    def parse(self, response):
        for archive in response.css('.column-content .textile > ul > li > a'):
            url = self.base_url + archive.css('::attr(href)').extract_first()

            yield scrapy.Request(url=url, callback=self.parse_archive)
    \end{minted}

Die Archivseite beinhaltet Links zu den Polizeitberichten. Außerdem wird hier jeder Bericht mit einer Kategorie/Ort
und einem Datum gekennzeichnet. Die Archivseite ist desweiterem in mehrere Seiten gegliedert. Pro Seite werden 50 Berichte angezeigt.
Durch extrahieren des "`Nächste Seite"'-Links und einem rekursiven Aufruf der Funktion können wir alle Seiten aus einem
Archiv crawlen. Auf jeder Seite werden die Links zu den Berichten extrahiert und als Request-Objekt in die Queue gelegt.

\begin{minted}[xleftmargin=20pt,linenos,breaklines,fontsize=\small]{python}
    def parse_archive(self, response):
        for report in response.css('.list-autoteaser > li.row-fluid'):
            timestamp = report.css('.date::text').extract_first()
            category = report.css('.category::text').extract_first()

            url = self.base_url + report.css('a::attr(href)').extract_first()

            yield scrapy.Request(
                url=url,
                callback=self.parse_report,
                meta={'timestamp': timestamp, 'category': category}
            )

        next_page_url = response.css('.html5-nav > div > ul > .pager-item-next > a::attr(href)').extract_first()

        if next_page_url:
            next_page_url = self.base_url + next_page_url

            yield scrapy.Request(url=next_page_url, callback=self.parse_archive)
    \end{minted}

\pagebreak

Als letztes gibt es die Seite des Polizeitberichts selbst.
Hier extrahieren wir den kompletten Text des Berichts über folgenden CSS-Selector.

\begin{minted}[xleftmargin=20pt,linenos,breaklines,fontsize=\small]{python}
    response.css(
        '.column-content > .article > .body .polizeimeldung::text,' +
        '.column-content > .article > .body p::text,' +
        '.column-content > .article > .body strong::text'
    ).extract()
    \end{minted}

Außerdem wird ein Item-Objekt mit den Daten des Berichts erstellt, welches anders als ein Request-Objekt behandelt wird.

\subsection{Item}

Bei Scrapy werden die extrahierten Daten in Items zusammengefasst. Hierzu dient eine einfache Klasse.
In unserem Fall haben wir ein Item für die Polizeiberichte definiert.

\begin{minted}[xleftmargin=20pt,linenos,breaklines,fontsize=\small]{python}
    class PoliceReportBerlinItem(scrapy.Item):
        # define the fields for your item here like:
        created = scrapy.Field()
        updated = scrapy.Field()
        timestamp = scrapy.Field()
        category = scrapy.Field()
        text = scrapy.Field()

        text_pre_processed_v1 = scrapy.Field()
        text_pre_processed_v2 = scrapy.Field()
        text_pre_processed_v3 = scrapy.Field()
        text_pre_processed_v4 = scrapy.Field()

        url = scrapy.Field()
        title = scrapy.Field()
\end{minted}

\subsection{Middleware}

Mit einer Middleware kann auf alle von Scrapy getätigten Requests eingegangen werden. Wir benutzen keine selbsterstelle
Middleware in unserem Projekt. Jedoch werden bei Scrapy einige Middlewares standardmäßig aktiviert, wie z.B.
die RetryMiddleware, welche einen Request ein weiteres Mal in die Queue aufnimmt, falls er fehlschlägt.

Für die Implementierung einer eigenen Middleware wären unterschiedliche Szenarien vorstellbar. Zum Beispiel könnte man
jeden Request mit bestimmten Cookies versehen, wenn die Seite einen Login erfordert. Für unseren Fall war das jedoch nicht
notwendig.

\subsection{Pipeline}

Jedes von Scrapy gecrawlte Item wird in einer definierten Reihenfolge durch alle definierten Pipelines gesendet.
In unserem Fall haben wir eine MongoDB Pipeline an letzter Stelle erstellt, die den Bericht in einer MongoDB-Datenbank abspeichert.

\section{Ergebnis}

Wir konnte erfolgreich einen Crawler implementieren, der alle archivierten Polizeimeldung der Berliner Polizei herunterläd
und in einer MongoDB speichert. Die Arbeit mit Scrapy stellte sich dabei als sehr komfortabel heraus.
Durch die vielen Einstellungsmöglichkeiten fiel es uns leicht, einen zuverlässigen Crawler zu realisieren.
Außerdem schlägt einem Scrapy direkt eine übersichtliche Datei und Ordnerstruktur vor. Dadaurch waren wir in der Lage, einen zuverlässigen Crawler
von hoher Codequalität und Erweiterbarkeit zu entwickeln.

Im Laufe des Projektes bekamen wir Fehlermeldung, dass wir zu viele Anfrangen gleichzeitig an den Server senden.
Unsere Anfrangen wurdem demnach abgelehnt. Durch die leichte Konfigurierbarkeit von Spacy konnten wir dieses Problem schnell lösen.
Es musste lediglich die Variable \emph{DOWNLOAD\_DELAY} auf den Wert 2 und die Variable \emph{CONCURRENT\_REQUESTS} auf den Wert 1
gesetzt werden, um ein Verzögerung zwischen den Seitenaufrufen zu erwirken.



