\chapter{Rust Crawler \small(Schmeling)}
\label{cha:rust}
Im folgenden Kapitel wird der Rust Crawler erläutert. Es werden im Kern der
Planungs- und Entwicklungsprozess beschrieben. Zunächst soll aber darauf
eingegangen werden, welche Motivation dahintersteckt, einen Crawler in einer
anderen Programmiersprache als der vorgesehenen Sprache Python\cite{python}
zu entwickeln, um was für eine Sprache es sich dabei handelt und welche Vor- und
Nachteile zu erwarten sind.


\section{Motivation}
\label{sec:rust_motivation}
Im Studium werden von den Studenten/innen in nahezu allen Veranstaltungen
Implementierungen der gelehrten Anwendungen und Systeme erarbeitet. Und in
nahezu jeder Veranstaltung werden die Technologien - das heißt die
Programmiersprachen, Frameworks oder Bibliotheken - vorgegeben. Da das
Einarbeiten in neue Technologien oft sehr zeitraubend ist, wird in vielen
Fällen auf dem vorhandenen Wissen der Studenten/innen aufgebaut. Leider wird
dadurch aber auch der Blick über den Tellerrand hinaus kaum gefördert. Umso
mehr Motivation entsteht dann, wenn den Studenten/innen die Möglichkeit
gegeben wird, die verwendeten Technologien selbst auszuwählen und im Rahmen
der Veranstaltung zu erforschen, um ein gegebenes Ziel zu erreichen.\\
Da im Rahmen des Veranstaltungskonzeptes das Verfolgen von eigenst gesetzten
Lernzielen im Mittelpunkt steht, bot es sich an, das Lernziel so zu
formulieren, dass das Einarbeiten in eine neue Programmiersprache Teil dessen
wird. So steht das Lernziel auf zwei zentralen Säulen:

\begin{itemize}
	\item Wie entwickele ich einen Crawler?
	\item Wie entwickele ich mit der Programmiersprache Rust\cite{rust}?
\end{itemize}

In den beiden folgenden Abschnitten soll nun darauf eingegangen werden, um
was für eine Programmiersprache es sich bei Rust handelt und warum diese für
den gegebenen Anwendungsfall eine mögliche Alternative zu Python darstellt.

\subsection{Die Programmiersprache Rust}
\label{sub:rust_sprache}
Rust ist eine Systemprogrammiersprache vergleichbar mit C/C++, ist aber im
Gegensatz eine sehr junge Programmiersprache. Version 1.0 erschien im Mai
2015. Rust weist einige Gemeinsamkeiten mit C/C++ auf:
\begin{itemize}
	\item minimale Runtime und kein Garbagecollector, dadurch als
	      Systemprogrammiersprache für beispielsweise Gerätetreiber, Embedded
	      Devices, Betriebssysteme einsetzbar
	\item Geschwindigkeit auf dem selben Level
	\item Kontrolle
	\item Generische Datentypen
\end{itemize}

Gleichzeitig besitzt Rust aber Merkmale, die in C/C++ nicht zu finden sind:
\begin{itemize}
	\item Kein manuelles Speichermanagement erforderlich
	\item Keine Segfaults
	\item Keine Dataraces
\end{itemize}

Das zentrale Modell, dass alle diese Eigenschaften erlaubt, ist das
Ownership-Modell\cite{ownership}. Jede Variable in Rust hat genau einen
Besitzer (Owner). Nur der Besitzer darf entscheiden, was mit einer Variable
passiert. Dabei gibt es drei mögliche Handlungen:
\begin{itemize}
	\item Besitzer gibt Variable weiter, verzichtet damit auf den Besitz und
	      kann die Variable nicht mehr verwendeten
	\item Besitzer leiht Variable aus (Referenz), die Variable kann nur
	      lesend vom Ausleiher verarbeitet werden. Es kann beliebig viele
	      gleichzeitige Ausleihen geben.
	\item Besitzer leiht Variable veränderlich aus (mutable Reference), die
	      Variable kann von dem Ausleiher lesend und schreibend verarbeitet werden.
	      Es kann aber nur eine Ausleihe zur gleichen Zeit geben. Es darf neben
	      einer veränderlichen Ausleihe auch keine unveränderlichen Ausleihen
	      geben.
\end{itemize}

Rust bietet einige Highlevel-Abstractions, wie man sie nur aus
Programmiersprachen mit Garbagecollection kennt, wodurch Rust sich angenehm
schreiben und lesen lässt. Zudem Rust erlaubt funktionale, objektorientierte
und auch nebenläufige Programmierung. Diese Eigenschaften zusammen mit der
Geschwindigkeit und der Speichersicherheit machen Rust auch für viele andere
Anwendungsbereiche interessant, wie beispielsweise Webserver,
Anwendungsentwicklung, Spiele, etc.\\
Zuletzt soll noch Cargo\cite{cargo} erwähnt werden, welches Rust's Projekt-,
Build- und Dependency-Management-Tool ist. Bibliotheken werden in Rust an
einem zentralen Ort - \url{https://www.crates.io}) - zur Verfügung gestellt.
Cargo und Rust-Bibliotheken verhalten sich vergleichbar zu NPM\cite{npm} und
Node\cite{node}-Modulen. Dadurch ist das Verwalten von Projekten, das managen
von Bibliotheken und das Testen und Kompilieren extrem einfach und
nutzerfreundlich.


\subsection{Crawler und Rust}
\label{sub:crawlrust}
Die folgende Liste stellt stichpunktartig dar, welche Eigenschaften von Rust
für die Entwicklung eines Crawlers sprechen.
\begin{itemize}
	\item Geschwindigkeit: hiermit ist die tatsächlich abrufbare
	      CPU-Geschwindigkeit gemeint, die beim Parsen von Dokumenten ein Vorteil
	      sein kann
	\item Concurrency: parallele Anfragen verarbeiten
	\item Einbinden vorhandener Bibliotheken aus anderen Programmiersprachen:
	      fehlende Rust-Bibliotheken können ersetzt werden
\end{itemize}


\section{Planung}
\label{sec:rust_planung}
Zunächst muss überlegt werden, welche Funktionalitäten für die
Implementierung des Crawlers benötigt werden. Dabei hilft aufzuzeigen, in
welchen Schritten ein Crawler arbeitet. Folgende Liste soll eine Schleife
darstellen, die wiederholt wird:
\begin{itemize}
	\item Url aus Url-Queue nehmen
	\item Get-Anfrage senden
	\item Responsebody verarbeiten
	      \begin{itemize}
		      \item Urls extrahieren und an Url-Queue anhängen
		      \item Polizeiberichte extrahieren und in Datenbank speichern
	      \end{itemize}
\end{itemize}

Aus diesen Einzelschritten ergeben sich die Funktionalitäten, die
implementiert werden müssen:
\begin{itemize}
	\item Urls in Datenstruktur speichern
	\item Http-Get-Anfragen senden, möglichst asynchron
	\item Html-Body gezielt auf Elemente, Klassen, IDs, etc. hin durchsuchen und
	      extrahieren, möglichst parallel
	\item Paralleles Verarbeiten der Html-Bodies
	\item Schnittstelle für Datenbank implementieren
	\item Zeitmessung für Diagnose und Vergleich
\end{itemize}

Für alle diese Funktionalitäten lassen sich Bibliotheken finden. Dabei sind
die Bibliotheken unabhängig vom gegebenen Anwendungsfall und unabhängig
untereinander. Jeder einzelne Schritt muss manuell implementiert werden, da
es bisher keinerlei Bibliothek oder Framework zum Crawlen gibt.\\
Folgende Liste enthält die Bibliotheken, die für die Umsetzung benötigt werden:
\begin{itemize}
	\item reqwest\cite{reqwest}: Http-Get-Anfragen senden
	\item select\cite{select}: Html-Dokumente extrahieren
	\item rayon\cite{rayon}: Parallelität
	\item tokio\cite{tokio}: Asynchronität
	\item chrono\cite{chrono}: Zeit messen
	\item mongodb\cite{mongodb}: Datenbankschnittstelle
\end{itemize}


\section{Implementierung}
\label{sec:rust_implementierung}
Zentrales Element des Crawlers ist eine Schleife, die folgende Logik wiederholt:
\begin{itemize}
	\item Anfragen an alle Urls in der Url-Queue senden
	\item Urls aus den Response-Bodies extrahieren und an die Url-Queue anhängen
	\item Polizeiberichte aus den Response-Bodies extrahieren und in die
	      Datenbank speichern
	\item Abfrage, ob es noch unbesuchte Urls in der Url-Queue gibt: wenn nein,
	      Programmende
\end{itemize}

Die genaue Arbeitsweise wird in den folgenden Unterkapiteln näher beschrieben.



\subsection{Url-Queue}
Die Url-Queue wurde als HashMap implementiert, die als Schlüssel die Url und
als Value ein Boolschen Wert enthält, der angibt, ob eine Url bereits
angefragt wurde. Durch die Implementierung als HashMap lässt sich sehr
einfach abfragen, ob eine bestimmte Url bereits in der Url-Queue vorhanden
ist. Dies ist von Bedeutung, da sich bestimmte Urls auf vielen Internetseiten
wiederfinden und es keine doppelten Urls in der Url-Queue geben soll ,da es sonst
zu Mehrfachanfragen an ein und dieselbe Url gibt.

\begin{minted}[xleftmargin=20pt,linenos,breaklines,fontsize=\small]{rust}
    let mut url_queue: HashMap<String, bool> = HashMap::new();
\end{minted}


\subsection{Http-Get-Anfragen}
Zu Beginn der Schleife wird an alle Urls in der Url-Queue, an die noch keine
Anfrage gesendet wurde, eine Get-Anfrage gesendet. Sollte es eine erfolgreiche
Antwort geben, dann wird der Html-Body in einer Liste gespeichert und die
zugehörige Url wird markiert als bereits angefragt.

\begin{minted}[xleftmargin=20pt,linenos,breaklines,fontsize=\small]{rust}
let mut bodies: Vec<String> = Vec::new();
for (url, state) in url_queue.iter_mut() {
    if *state == false {
        match request::get(&url) {
            Some(res) => {
                bodies.push(res);
                *state = true;
            }
            None => (),
        }
    }
}
\end{minted}

Für die eigentliche Get-Anfrage wird folgende Funktion genutzt:

\begin{minted}[xleftmargin=20pt,linenos,breaklines,fontsize=\small]{rust}
pub fn get(url: &str) -> Option<String> {
    let body = reqwest::get(url);
    if body.is_err() {
        return None;
    }

    let mut body = body.unwrap();
    if !body.status().is_success() {
        return None;
    }

    let body = body.text();
    if body.is_err() {
        return None;
    }

    let html = body.unwrap();

    Some(html)
}
\end{minted}

Die Methode gibt dabei im Fall des Erfolges einen String zurück (den
Html-Body), im Falle eines Misserfolges gibt sie nichts zurück
(strenggenommen gibt sie ein \emph{None} zurück, was einem \emph{Null} in
anderen Programmiersprachen ähnelt).\\
Anmerkung: sämtliche Get-Anfragen des Crawlers geschehen synchron. Leider war es
nicht mehr möglich, die asynchrone Funktionalität zu implementieren. Mehr dazu
in Kapitel \ref{sec:rust_ergebnis}.


\subsection{Urls extrahieren}
Nun wird über alle Html-Bodies iteriert und aus jedem Body werden Urls
extrahiert. Alle Urls, die noch nicht in der Url-Queue sind, werden an diese
angehängt.

\begin{minted}[xleftmargin=20pt,linenos,breaklines,fontsize=\small]{rust}
for body in bodies.iter() {
    let urls = url::from_html(&body);
    for url in urls {
        if !url_queue.contains_key(&url) {
            url_queue.insert(String::from(url), false);
        }
    }
}
\end{minted}

Für das Extrahieren der Urls aus den Bodies wird folgende Funktion verwendet:

\begin{minted}[xleftmargin=20pt,linenos,breaklines,fontsize=\small]{rust}
pub fn from_html(html: &str) -> Vec<String> {
    let mut urls: Vec<String> = Vec::new();
    let document = Document::from(html);

    for elem in document.find(Name("a")) {
        match elem.attr("href") {
            Some(url) => {
                let mut url = String::from(url);
                if url.starts_with("/") {
                    url.insert_str(0, "https://www.berlin.de");
                }
                if url.starts_with("https://www.berlin.de/polizei/polizeimeldungen/archiv/20")
                    && url.contains("page_at")
                {
                    urls.push(url);
                }
            }
            None => (),
        }
    }

    urls
}
\end{minted}

Mit Hilfe der Bibliothek \emph{select} werden alle \emph{a}-Elemente gesucht
und der Wert des Attributs \emph{href} extrahiert. Da einige Urls nur als
Subdomain angegeben sind, müssen diese noch aufgefüllt werden. Und zuletzt
werden nur die Urls in die Liste getan, die die Archiv-Seiten adressieren.


\subsection{Polizeiberichte extrahieren}
Nun wird über alle Html-Bodies iteriert und in jedem Body wird nach
Polizeiberichten gesucht. Anschließend werden die Berichte in der Datenbank
gespeichert.

\begin{minted}[xleftmargin=20pt,linenos,breaklines,fontsize=\small]{rust}
let reports: Vec<Report> = bodies
    .iter()
    .flat_map(|b| Report::from_html(b))
    .collect();

for report in reports {
    mongo.add(report);
}
\end{minted}

Für das Finden der Polizeiberichte wird folgende Funktion verwendet:

\begin{minted}[xleftmargin=20pt,linenos,breaklines,fontsize=\small]{rust}
pub fn from_html(html: &str) -> Vec<Report> {
    let mut reports: Vec<Report> = Vec::new();

    let document = Document::from(html);
    let li_elements: Vec<select::node::Node> = Vec::new();
    for elem in document.find(Class("list-autoteaser").descendant(Class("row-fluid"))) {
        let title = Report::parse_title(elem);
        let url = Report::parse_url(elem);
        let location = Report::parse_location(elem);
        let date = Report::parse_date(elem);
        let text = Report::parse_text(&url);

        if text.len() > 0 {
            reports.push(Report::new(title, url, location, date, text));
        }
    }

    reports
}
\end{minted}

Dabei wird wieder die Bibliothek \emph{select} genutzt, um gezielt Html-Elemente
und deren Nachfahren zu finden und Attribute auszulesen. Nach welchen
Elementen gesucht werden muss, wurde vorher manuell aus Quelltexten einiger
Internetseiten, die einen Polizeibericht enthalten, herausgelesen.\\
Zum Parsen der gesuchten Attribute Titel, Url, Ort, Datum und Uhrzeit sowie
dem eigentlichen Text wurden Helfer-Funktionen genutzt. Diese funktionieren
aber analog zu obiger Funktion und werden hier daher nicht abgedruckt.

\section{Ergebnis}
\label{sec:rust_ergebnis}
Es wurde erfolgreich ein Crawler in der Programmiersprache Rust entwickelt,
der die Polizeiberichte der Berliner Polizei crawlt und in einer Datenbank
speichert. Das Ziel, einen Crawler zu implementieren, wurde erreicht. Das
Ziel, in der Programmiersprache Rust zu entwickeln, wurde zwar auch erfüllt,
jedoch nicht zu meiner vollen Zufriedenheit. Durch die geringen
Vorerfahrungen mit Rust war es teils sehr mühselig, selbst einfache Dinge zu
implementieren. Es hat sich gezeigt, was ich vorher durchaus vermutet habe: Rust
ist keine einfach zu erlernende Programmiersprache. Ich bin meinem Lernziel
dabei trotzdem nah gekommen, da ich im Lauf der Entwicklung wirklich viel
über Rust lernen konnte. Da das zu einem großen Teil aber an gemachten
Fehlern und zeitintensiver, teils frustrierender Fehlersuche lag, war die
Erfahrung dabei durchwachsen.\\
Grundsätzlich ist die Dokumentation der genutzten Bibliotheken gut. Jedoch
wird ein gewisses Maß an Erfahrung im Umgang mit Rust
vorausgesetzt. Das war ein Faktor, weshalb das Arbeiten mit einigen
Bibliotheken recht mühsam war. Da Rust zudem eine sehr junge
Programmiersprache ist, gibt es verhältnismäßig wenige Beispiele dafür zu
finden, wie gewisse Problem gelöst werden können. Ich musste
unverhätlnismäßig viel Zeit in die Recherche der Funktionsweise von
Bibliotheken, unter anderen auch der Standardbibliothek, investieren.\\
Leider gelang es mir trotz intensiver Recherche nicht, die Bibliothek
\emph{tokio} mit einzubinden, weshalb asynchrone Get-Anfragen nicht mehr
implementiert werden konnten. Dabei habe ich aber auch die Unterschiede und
Einsatzgebiete von OS-Threads, Green-Threads and Async I/O lernen können.
Grundsätzlich wäre die Implementierung von OS-Threads in Rust relativ einfach
- bedingt durch die Low-Level-Eigenschaften von Rust. Jedoch sind diese für
asynchrone Anfragen fehl am Platz. Hier würden sich Green-Threads anbieten
oder eben Async I/O, wobei \emph{tokio} als Referenz-Bibliothek für Async I/O
in Rust gilt. Allerdings habe ich nicht nur durch Erfahrung, sondern auch
durch Recherche festgestellt, dass ich bei weitem nicht der/die Einzige bin
mit Problemen beim Verständnis der Funktionsweise von \emph{tokio}.

\subsection{Verbesserungsmöglichkeiten}
Als zentraler Punkt sei hier nochmal \emph{tokio} und Async I/O genannt.
Derzeit arbeitet das Rust-Entwicklerteam an der Weiterentwicklung und
Überarbeitung von asynchronen Funktionalitäten, teilweise sollen diese mit in
die Standardbibliothek aufgenommen werden. Erste Beispiele zeigen, dass das
händeln von asynchronen Abläufen damit deutlich vereinfacht werden wird.
Leider war und ist bisher aber noch kein Stand erreicht, der das neue
Async-Modell benutzbar macht. Das Rust-Entwicklerteam rechnet damit, dies in
den kommenden Monaten zu vervollständigen. Der Grundtenor in der
Rust-Community ist derzeit jener, sich als Beginner noch nicht mit Async I/O
auseinanderzusetzen, sondern zu warten, bis das neue Async-Modul fertig ist,
da es dadurch enorme Änderungen und Erleichterungen in der Implementierung
gibt.\\
Als weitere Verbesserungsmöglichkeit sehe ich allgemein die Implementierung
in Rust. Zwar funktioniert der Crawler wie gewünscht, jedoch ist mein
Eindruck vom Programm der, dass sowohl der Aufbau bzw. die Struktur als auch
viele Stellen im Quelltext nicht dem entsprechen, was mit Rust möglich wäre
und was man in Rust als idiomatisch ansieht. Hier fehlt mir einfach die
Erfahrung im Ausreizen der Möglichkeiten, die Rust zur Verfügung stellt. Der
Quelltext sieht unschön aus.