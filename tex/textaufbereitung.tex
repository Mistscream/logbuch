\chapter{Textaufbereitung (Sebastian Jüngling)}

Um die Texte der Berliner Polizeiberichte in späteren Arbeitsschritten für die Implementierung einer Suchmaschine verwenden zu können, ist es nötig diese entsprechend aufzubereiten. Das bedeutet im Grunde, dass die Fließtexte auf ihre wesentlichen bedeutungstragenden Begriffe reduziert werden. 
\\Für diesen Arbeitsschritt haben wir uns für die Verwendung der mächtigen NLP Python Library SpaCy entschieden. 

\section{SpaCy}
Die Open-Source Library SpaCy ermöglicht fortgeschrittenes Natural Language Processing für mehrere Sprachen, unter Anderem der deutschen. SpaCy bietet hierfür eine Vielzahl verschiedener Features, wie z.B. Tokenization, Lammatization, Part-of-Speech Tagging, dependency parsing und demenentsprechend besonders wichtige Features, wie das Erkennen von Noun-Phrases (Nominalphrasen) und Named-Entitities (Textelemente vordefinierter Kategorien). 

\subsection{Umsetzung}
Im folgenden wird auf die grundsätzliche Funktionsweise der implementierten Textaufbereitung unter der Verwendung von SpaCy eingegangen.

\subsection{Ergebnis}





